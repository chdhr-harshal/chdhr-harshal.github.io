\documentclass[10pt]{moderncv}
\usepackage {marvosym}
\usepackage[T1]{fontenc}
\moderncvtheme[green]{casual}

\usepackage[utf8]{inputenc}

\usepackage[margin=0.5in]{geometry}
\setlength{\hintscolumnwidth}{2cm}
\AtBeginDocument{\recomputelengths}

\firstname{Harshal }
\familyname{Chaudhari}
\nopagenumbers{}
\newcommand{\LinkTo}[2]{\hfill \Letter \href{#1}{ \textit{#2}}}


\begin{document}
\maketitle

\section{Personal Information}
\cvline{Date of Birth}{May 14, 1991}
\cvline{Contact}{\email{harshal@bu.edu}{harshal@bu.edu}\ \ \ \ \ \ \ \ +1-857-294-8568}
\cvline{Website}{\href{https://www.harshalc.com}{https://www.harshalc.com}}
\cvline{Github}{\href{https://github.com/chdhr-harshal}{https://github.com/chdhr-harshal}}

\section{Research Interests}
\cvline{}{Explainable A.I. Systems, Robust Optimization, Reinforcement Learning.}

\section{Education}
\cventry{2015--2021}{Ph.D. in Computer Science}{Boston University}{}{}{Research Advisers: Prof. John Byers, Prof. Evimaria Terzi}
\cventry{2013--2015}{M.S. in Computer Science}{Boston University}{}{}{}
\cventry{2009--2013}{B.E.(Hons.) in Computer Science}{Birla Institute of Technology and Science, Pilani}{}{}{}

\section{Work Experience}
\cventry{2023--current}{Senior Applied Scientist II}{Etsy, Boston, USA}{}{}
{
    Developed state of the art A.I. models using real time user activity to guide recommendations for users on various 
        Etsy platforms. 
    Involved in guiding junior scientists on developing ML models alongside interviewing and planning initiatives to 
        power next-gen recommender systems at Etsy.
}

\cventry{2021--2023}{Senior Applied Scientist I}{Etsy, Boston, USA}{}{}
{
    As a member of Personalized recommendations ranking team, I work on developing fair, unbiased and scalable 
        recommender system algorithms.
    Ranking models developed by me power recommendations across various sections of Etsy website, iOS and Android 
        smartphone applications.
}

\cventry{2018--2021}{Applied Scientist (part-time)}{Zillow Group, Seattle, USA}{}{}
{
    Research collaboration with the AI Relevance team, we explore the problem of developing a low-latency and
        high-quality personalized recommendation system that is robust to data imperfections while simultaneously 
        ensuring fairness for protected groups of consumers.
}

\cventry{2018}{Applied Scientist Intern}{Zillow Group, Seattle, USA}{}{}
{
    Developed an unsupervised scalable framework to identify the state of user in their home-buying journey based upon 
        their interaction history on the real-estate marketplace Zillow. 
    Injecting the features derived from identification of the state of journey of a user into the personalized 
        recommendation platform at Zillow results in a significant improvement on the key metrics.
}

\cventry{2015}{Data Science Intern}{Amplero, Seattle, USA}{}{}
{
    Developed a user simulator for a personalized marketing platform, Amplero, developed by Globys Inc. 
    The characteristics of the simulated users are derived from real world telecoms usage data in a probabilistic 
        manner. 
    It removes operational lag associated with marketing, facilitates A/B testing for various predictive models 
        devised by Amplero. 
    Modular design of the user simulator further enables fine tuned differential analysis of the strategies.
}

\cventry{2013}{Research Intern}{Siemens Corporate Research, Bangalore, India}{}{}
{
    Developed a clone prioritization algorithm for identification of code clones and optimal resource allocation for 
        clone refactoring. 
    Modeled this as a multi-constrained, multi-objective Knapsack problem and investigated various heuristics in 
        multi-criterion branch and bound algorithms in addition to evolutionary algorithms based on Pareto optimality. 
    We developed an Eclipse plug-in for the assessment of code duplication characteristics.
}

\section{Research Experience}
\cventry{2023}{Towards Flexibility and Robustness of LSM Trees}{}{}{VLDB Journal: Special Issue on ML and DB}
{
    /textit{Co-authors: Andy Huynh, Prof. Evimaria Terzi, Prof. Manos Athanassoulis}\\
    Log-Structured Merge trees (LSM trees) are increasingly used as the storage engines behind several data systems,
        frequently deployed in the cloud. 
    Operating in a shared infrastructure like the cloud comes with workload uncertainty due to fast evolving nature of 
        modern applications. 
    Systems with static tuning discount variability of workloads and provide inconsistent and suboptimal performance. 
    Building upon our previous work, Endure, we introduce flexible compaction policy, viz., K-LSM -- that allows us to 
        express fine-grained hybrid compaction policies between the popular leveling and tiering policies. 
    With exhaustive experimentation, we analyze performance improvements offered by K-LSM policy over state-of-the-art 
        baselines.
}

\cventry{2022}{Endure: A Robust Tuning Paradigm for LSM Trees Under Workload Uncertainty}{}{}{VLDB 2022}
{
    /textit{Co-authors: Andy Huynh, Prof. Evimaria Terzi, Prof. Manos Athanassoulis}\\
    Modern LSM-tree backed key-value stores co-tune merge policies, buffer sizes and the false positive rates for the 
        Bloom filters across different levels of LSM-tree. 
    These systems typically maximize throughput associated with updates, point and range lookup queries for fixed 
        expected workloads. 
    However, the analytically obtained optimal design-parameters for these systems are not always feasible. 
    In this work, we introduce Endure -- a new paradigm for tuning LSM trees in the presence of workload uncertainty. 
    Robust tunings output by Endure lead up to a 5x improvement in throughput.
}

\cventry{2021}{Fleet Management Strategies for Urban Mobility-on-Demand Systems}{}{}{Ph.D. Thesis}
{
    In recent years, the paradigm of personal urban mobility has radically evolved as an increasing number of 
        Mobility-on-Demand (MoD) systems continue to revolutionize urban transportation. 
    Hailed as the future of sustainable transportation, with significant implications on urban planning, these systems 
        typically utilize a fleet of shared vehicles such as bikes, electric scooters, cars, etc., and provide a 
        centralized matching platform to deliver point-to-point mobility to passengers. 
    In this dissertation, we study MoD systems along three operational directions—
    (1) modeling: developing analytical models that capture the rich stochasticity of passenger demand and its impact 
        on the fleet distribution,
    (2) economics: devising strategies to maximize revenue, and 
    (3) control: developing coordination mechanisms aimed at optimizing platform throughput.
}

\cventry{2020}{A General Framework for Fairness in Multistakeholder Recommendations}{}{}{FAccTREC, RecSys 2020}
{
    /textit{Co-authors: Sangdi Lin, Ondrej Linda}\\
    Traditionally, multistakeholder recommendations problems have been formulated as integer linear programs which 
        compute recommendations in an offline fashion, by incorporating provider constraints.
    Such approaches can lead to unforeseen biases wherein certain users consistently receive low utility 
        recommendations in order to meet the global provider coverage constraints. 
    We propose a submodular optimization based framework incorporating seller coverage objectives alongside user 
        objectives in a real-time personalized recommender system.
}

\cventry{2020}{Learn to Earn: Enabling Coordination Within a Ride-Hailing Fleet}{}{}{IEEE BigData 2020}
{
    /textit{Co-authors: Prof. John Byers, Prof. Evimaria Terzi}\\
    In this work, we explore the problem of maximizing earnings of drivers employed by ride-hailing platforms like 
        Uber, Lyft, etc. Our work confirms the idea that even in a high-dimensional and big-data domain such as 
        ride-hailing, the inherent structure of the data can be leveraged to develop a simple, interpretable, fair and 
        highly efficient framework that aims to achieve this goal. 
    Furthermore, we provide evidence for model robustness and generalizability using large-scale simulations based on 
        publicly avilable New York City taxi datasets.
}

\cventry{2018}{Markov Chain Monitoring}{}{}{SDM 2018}
{
    /textit{Co-authors: Prof. Michael Mathioudakis, Prof. Evimaria Terzi}\\
    Given an initial distribution of items over the nodes of a Markov chain, we wish to estimate the distribution of 
        items at subsequent times. In deriving these estimates, we issue queries to retrieve partial information on the 
        distribution of items. 
    For different types of queries, we design efficient algorithms for picking the right queries that make our 
        estimates as accurate as possible.
}

\cventry{2018}{Putting Data in Driver's Seat: Optimizing Earnings for On-Demand Ride-Hailing}{}{}{WSDM 2018}
{
    /textit{Co-authors: Prof. John Byers, Prof. Evimaria Terzi}\\
    In this study, we model the passenger seeking behavior of the Uber drivers as a controlled Markov Decision Process 
        (MDP) over a finite horizon. 
    The parameters of this MDP are set using Uber Rider API and publicly available New York Taxi datasets.
    Using this model, we devise three optimal strategies for Uber drivers and evaluate them over multiple simulations 
        of MDP. 
    We provide a sensitivity analysis to account for uncertainties in the MDP parameters.
}

\cventry{2017}{Impacts of free app promotion: A case study on Amazon Appstore}{}{}{WCBA 2017, TSMO 2018}
{
    /textit{Co-authors: Prof. John Byers}\\
    In this study, we investigate the longer-term consequences of free app promotions on the performance of apps on 
        Amazon Appstore. 
    In particular, we quantify the causal impact of such promotions on apps’ future download volumes, star ratings, and 
        sales rank using a multi-level model. 
    In addition, we show the presence of a cross-market spillover effect of such promotions on the performance of the 
        same apps on Google Playstore. 
    Our results underscore a nuanced set of trade-offs for an app developer: do the benefits of running a promotion and 
        boosting ones’ sales rank warrant the lost revenue and risk of lower user ratings in the long run?
}

\section{Professional Service}
\cvline{PC Member}{WSDM `24, NeurIPS `23, SDM `23, KDD `23, WSDM `23, KDD `21, KDD `20, KDD `19, WWW `18.}
\cvline{Reviewer}{ECML-PKDD `19, TKDE `18 (Journals).}
\cvline{}{ICML `23, ICML `22, ICML`21, ICML `20, WSDM `18, ICDE `18, WWW `17 and ICWSM `17 (Conferences).}

\section{Teaching Experience}
\cvline{CS565}{\textbf{Algorithmic Data Mining}, \textit{Instructor: Prof. Evimaria Terzi.}}
\cvline{CS131}{\textbf{Combinatoric Structures in Discrete Mathematics}, \textit{Instructor: John Byers.}}
\cvline{CS111}{\textbf{Introduction to Java Programming}, \textit{Instructor: Prof. David Sullivan.}}

\section{Awards and Scholarships}
\cvline{2007--2012}{National Talent Search Scholarship by the Government of India.}
\cvline{2005--2007}{Maharashtra Talent Search Scholarship by the Government of Maharashtra state.}
\cvline{2005--2007}{Bombay Talent Search Scholarship.}

\section{Programming Experience}
{
	\cvline{}{
    	\begin{tabular*}{0.8\textwidth}{@{\extracolsep{\fill}} l p{0.5\textwidth}}
		Programming&{\small Tensorflow, PyTorch, Apache Spark, Python, Scala, Java, R, C/C++, SQL, etc.}\\
    	Cloud Infrastructure&{\small GCP, AWS}\\
    	\end{tabular*}
    }	
}

\end{document}
